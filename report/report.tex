%% The first command in your LaTeX source must be the \documentclass command.
\documentclass[acmconf]{acmart}
% % Packages
%\usepackage{cite}
\usepackage{listings}
\usepackage{amsmath,amssymb,amsfonts}
\usepackage{algorithmic}
\usepackage{graphicx}
\usepackage{textcomp}
\usepackage{xcolor}
\usepackage{booktabs}
\usepackage{parskip}
\setlength{\parskip}{1em}
\usepackage[bottom]{footmisc}
\usepackage{url}
\usepackage{enumitem}
\usepackage{hyperref}
\usepackage{totpages}

%% \BibTeX command to typeset BibTeX logo in the docs
\AtBeginDocument{%
  \providecommand\BibTeX{{%
    \normalfont B\kern-0.5em{\scshape i\kern-0.25em b}\kern-0.8em\TeX}}}

%% Rights management information.  This information is sent to you
%% when you complete the rights form.  These commands have SAMPLE
%% values in them; it is your responsibility as an author to replace
%% the commands and values with those provided to you when you
%% complete the rights form.
% \setcopyright{acmcopyright}
% \copyrightyear{2018}
% \acmYear{2018}
% \acmDOI{10.1145/1122445.1122456}

% %% These commands are for a PROCEEDINGS abstract or paper.
% \acmConference[Woodstock '18]{Woodstock '18: ACM Symposium on Neural
%   Gaze Detection}{June 03--05, 2018}{Woodstock, NY}
% \acmBooktitle{Woodstock '18: ACM Symposium on Neural Gaze Detection,
%   June 03--05, 2018, Woodstock, NY}
% \acmPrice{15.00}
% \acmISBN{978-1-4503-XXXX-X/18/06}



%%
%% The majority of ACM publications use numbered citations and
%% references.  The command \citestyle{authoryear} switches to the
%% "author year" style.
%%
%% If you are preparing content for an event
%% sponsored by ACM SIGGRAPH, you must use the "author year" style of
%% citations and references.
%% Uncommenting
%% the next command will enable that style.
%%\citestyle{acmauthoryear}

%%
%% end of the preamble, start of the body of the document source.
\begin{document}

% import custom commands found in commands.tex
\newcommand{\fix}[1]{\textcolor{red}{#1}}
\newcommand{\derek}[1] {\textcolor{blue}{\textbf{[Derek: #1]}}}
\newcommand{\neha}[1] {\textcolor{green}{\textbf{[Neha: #1]}}}
\newcommand{\keanu}[1] {\textcolor{orange}{\textbf{[Keanu: #1]}}}
\newcommand{\manish}[1] {\textcolor{cyan}{\textbf{[Manish: #1]}}}
% set up commands to format RQ nicely
\newlist{questions}{enumerate}{2}
\setlist[questions,1]{label=RQ\arabic*.,ref=RQ\arabic*}
\setlist[questions,2]{label=(\alph*),ref=\thequestionsi(\alph*)}

%%
%% The "title" command has an optional parameter,
%% allowing the author to define a "short title" to be used in page headers.
\title[Survival Analysis of Open Source Projects]{Two Differing Approaches to Survival Analysis of Open Source Python Projects}

%%
%% The "author" command and its associated commands are used to define
%% the authors and their affiliations.
\author{Derek Robinson}
\email{drobinson@uvic.ca}
\author{Keanelek Enns}
\email{keanelekenns@uvic.ca}
\author{Neha Koulecar}
\email{nehakoulecar@uvic.ca}
\author{Manish Sihag}
\email{manishsihag@uvic.ca}
\affiliation{%
  \institution{\\University of Victoria}
  \department{Computer Science}
  \streetaddress{PO Box 1700 STN CSC}
  \city{Victoria}
  \state{British Columbia}
  \country{Canada}
  \postcode{V8W 2Y2}
}

%%
%% By default, the full list of authors will be used in the page
%% headers. Often, this list is too long, and will overlap
%% other information printed in the page headers. This command allows
%% the author to define a more concise list
%% of authors' names for this purpose.
\renewcommand{\shortauthors}{D. Robinson, K. Enns, N. Koulecar, M. Sihag}

%%
%% The abstract is a short summary of the work to be presented in the
%% article.
\begin{abstract}
% Pending
\end{abstract}

%%
%% The code below is generated by the tool at http://dl.acm.org/ccs.cfm.
%%
\begin{CCSXML}
<ccs2012>
<concept>
<concept_id>10002951.10003227.10003351</concept_id>
<concept_desc>Information systems~Data mining</concept_desc>
<concept_significance>500</concept_significance>
</concept>
</ccs2012>
\end{CCSXML}

\ccsdesc[500]{Information systems~Data mining}

%%
%% Keywords. The author(s) should pick words that accurately describe
%% the work being presented. Separate the keywords with commas.
\keywords{data science, survival analysis, open source, python, Kaplan Meier, Cox proportional hazards model, Bayesian analysis}


%%
%% This command processes the author and affiliation and title
%% information and builds the first part of the formatted document.
\maketitle

\section{Introduction}
The developers of Open Source Software (OSS) projects are often part of decentralized and geographically distributed teams of volunteers.
As these developers volunteer their free time to build such OSS projects, they likely want to be confident that the projects they work on will not become inactive.
If OSS developers are aware of key attributes that are associated with long-lasting projects, they can make informed assessments of a given project before devoting their time to it or they can strive to make their own projects exhibit those attributes.
Understanding which attributes of an OSS project lead to its longevity is what motivated Ali \emph{et al.} to apply survival analysis techniques commonly found in biostatistics to study the probability of survival for popular OSS Python projects \cite{ali2020cheating}.
Ali \emph{et al.} specifically studied the effect of the following attributes on the survival of OSS Python projects: publishing major releases, the use of multiple repositories, the type of version control system (VCS), and the size of the volunteer team.

Survival analysis is a set of methods used to determine how long an entity will live (or the time to a given event) and is most often used in the medical field.
\keanu{do we need to find a citation for this claim/definition?}
For example, survival analysis techniques can determine the probability that a patient will survive when given a certain treatment. 
Ali \emph{et al.} use a frequentist approach to survival analysis with methods including the Kaplan-Meier (K-M) survival estimator and the Cox Proportional-Hazards model.
Though there are advantages to using such approaches \keanu{list some advantages and cite}, another approach to survival analysis, Bayesian analysis, has its own set of advantages, namely \keanu{find the advantages, list them, cite}.

The authors of this paper resonate with Ali \emph{et al.'s} motivation. 
This paper serves as a replication of \cite{ali2020cheating} and seeks to determine if there are gaps or shortcomings in their analysis. 
This replication also provides artifacts so that others may see how the study was conducted and reproduce it with ease. 
In addition to the replication, this paper analyzes the same data set using a Bayesian approach to survival analysis as outlined in \cite{kelter2020bayesian} and seeks to compare the results of frequentist and bayesian approaches in the same domain.
Thus, the research questions this paper answers are as follows:

\begin{questions}
    \item How do major releases, the use of multiple repositories, the type of VCS, and the size of the volunteer team affect the survival of an OSS Python project?
    \item Are there any shortcomings or gaps in the study done by Ali \emph{et al.}?
    \item How do the findings of frequentist survival analysis differ from Bayesian survival analysis?
\end{questions}

The following section describes the methods used for both the replication study and the bayesian analysis study.
Section 3 shows the results for each analysis.
Section 4 is a discussion about the limitations of the study, the differences between the types of analysis performed, and suggestions about future work in the topic.
The final section concludes by summarizing the purpose and findings of this paper. 

\section{Methods}

Most of the work done in this study has been documented in a repository for the benefit of the reader.
A reference to this repository \keanu{and more?} can be found in appendix A below. \keanu{define artifacts?}

\subsection{Replication}

\keanu{I'm having a hard time figuring out what Neil would want here. I don't want to just repeat what the original study said, but I'm not sure what else to do}
\derek{We need to to talk about the same topcis that Ali et al. did. 
However, we need to use our own language. 
Maybe finding another SE replication paper and basing it off how they talk about the same topics as the original?}

Steps in the replication:

Importing data from Software Heritage Graph \cite{pietri2019software}

Inspecting the provided schema

Crafting queries to give us the necessary information (join necessary tables and group by appropriate fields, filter out dates that were out of range, create indicator field for censoring etc.)

Constructing additional fields (multi\_repo and duration) using pandas in python. Sort by duration and plot project from start to end month to recreate figure 1.

Perform frequentist analysis on data using survival package in R. First apply kaplan meier based on each attribute, then run Cox proportional hazards tool on the data.

\subsection{Bayesian Survival Analysis}

\keanu{I don't have anything to write for this}

\section{Results}

\subsection{Replication}

We found the results of our replication study to be extremely similar to those shown in the original paper.

\keanu{Put images here}

\keanu{talk about significance and meaning of the KM and Cox results}

\subsection{Bayesian Survival Analysis}

\keanu{Nothing for this yet}

\section{Discussion}

\subsection{Comparison of Analyses}

\subsection{Limitations}

\subsubsection{Limitations of the Methods}

The method Ali \emph{et al.} used for censoring was slightly ambiguous.

\subsubsection{Limitations of the Data}

There was no clear method for determining whether a project was hosted in multiple repositories.
The only unique identifier for each project was a url, from which the project name was extracted using regular expressions.
This may have resulted in the extraction of inconsistent project names across host types.
The assumption is that each project that was hosted on multiple repositories would be given the exact same name, and that projects of the same name hosted on different platforms were indeed the same project (which may not always be the case).
Additionally, it is possible for users on Github to give their projects the exact same name as pre-existing projects created by other users.
However, this issue was accounted for in our analysis, but it is unclear whether Ali \emph{et al.} accounted for this.

The data contains a large portion of censored data.
This means that we did not observe the abandonment of most of the projects.
As data points are censored (denoted by the vertical tick marks in the KM curves), we have a smaller and smaller group of data points to study.
This means that the results towards the 165 month mark may be less representative.
This is to be expected with any SE study, as recent years have lead to an exponential increase in the number of OSS projects \keanu{find citation?}.


\subsubsection{Replication Challenges}

\subsection{Future Work}

Increase the time frame of the study (the original paper could not do this because the paper was written in 2018).

\section{Conclusion}

%%
%% The next two lines define the bibliography style to be used, and
%% the bibliography file.
\bibliographystyle{ACM-Reference-Format}
\bibliography{refs.bib}

%%
%% If your work has an appendix, this is the place to put it.
\appendix

\section{Artifacts}
Project Repository: \url{https://github.com/DerekRobin/CSC578B-Project}

\end{document}
\endinput
%%
%% End of file `sample-manuscript.tex'.